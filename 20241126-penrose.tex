\documentclass{beamer}
\mode<presentation>
\usepackage{hyperref}
\usepackage{pdfpages}
\usepackage{siunitx}
\usepackage{tikz}
\usepackage{xcolor}
\usepackage{physics}
\usetikzlibrary{shapes.geometric,angles,quotes}

\definecolor{pGreen}{HTML}{006633}
\definecolor{sgbGreen1}{HTML}{66cc99}
\definecolor{sgbGreen2}{HTML}{ccffcc}

\definecolor{pRed}{HTML}{cc0033}
\definecolor{sgbRed}{HTML}{cc6688}
\definecolor{sgbRed2}{HTML}{FF99CC}

\definecolor{sgbGrey}{HTML}{333333}
\definecolor{sgbGrey2}{HTML}{996666}
\definecolor{sgbGrey3}{HTML}{999966}

\title{Exploring Modern Mathematics Through the Art of Tiling}
\author[dishajk]{Disha Kuzhively\\ \texttt{disha.jk@icts.res.in}}
\institute{International Centre for Theoretical Sciences - TIFR, Bangalore}
\date[Sci560]{Science Gallery Bangalore}
\begin{document}
\begin{frame}
    \titlepage
\end{frame}
\begin{frame}
    \frametitle{Outline}
    \tableofcontents[pausesections]
\end{frame}
\section{What is Tiling?}
\begin{frame}
    \frametitle{Tiling}
    \begin{block}{What is Tiling?}    
    Tiling is the covering of a surface using geometric shapes while ensuring there are \alert{no overlaps} and \alert{no gaps}.
    \end{block}
    \pause
\only<2>{    \tikz \node[draw] at (0,0) {Picture of tiled floor/wall};

    \tikz \node[draw] at (0,0) {Picture of checked fabric};

    \tikz \node[draw] at (0,0) {Picture of folded origami tesselation};}
    \only<3>{\tikz \node[draw] at (0,0) {Tiles with curved edges};}    
\end{frame}
\begin{frame}
    \frametitle{Tiling with Convex Regular Polygons}
    \only<1>{A convex regular polygon has $n$ straight edges and $n$ equal internal angles. It is convex because any line segment joining two vertices passes entirely through the polygon's interior.}
    
    \only<2>{Can you tile with convex regular polygons?} 
    
    \only<3-4,11-13>{How many \alert{edge to edge} tilings are possible using \alert{only one} type of tile?}

    \only<14->{How many \alert{edge to edge} tilings are possible using \alert{more than one} type of tile?}

    \only<1-3>{\tikz {\foreach \a in {3,...,6}{
        \node[regular polygon, regular polygon sides=\a, fill=sgbGreen2, draw, minimum size=2.25cm] at (\a*2.25,0) {};
        \node[regular polygon, regular polygon sides={\a+3}, draw, fill=sgbGreen2, minimum size=2.25cm] at (\a*2.25,-2.5) {};
        }}}
    
    \only<4>{
        \tikz \node[draw] at (0,0) {edge to edge tiling example};
        \tikz \node[draw] at (0,0) {not edge to edge tiling example};}
    
    \only<5-10>{Internal angle of an $n$-sided polygon}
    
    \only<19>{Consider $m=3$}
    
    \begin{columns}
        \begin{column}{0.4\textwidth}
            \only<5>{
                \tikz \node[regular polygon, regular polygon sides={11}, draw=sgbRed, minimum size=5cm] at (0,0) {11-gon};
            }

            \only<6>{
                \tikz {\node[regular polygon, regular polygon sides={11}, draw=sgbRed, minimum size=5cm] at (0,0) {};
                \foreach \angle in {0,...,10}{
                \draw (0,0) -- ({90+\angle*360/11}:2.5);}
                }
            }

            \only<7>{
                \tikz{\node[regular polygon, regular polygon sides={11}, draw=sgbRed, minimum size=5cm] at (0,0) {};
                \foreach \angle in {0,...,10}{
                    \draw (0,0) -- ({90+\angle*360/11}:2.5);
                    }
                \filldraw[fill=sgbGreen2,draw=black] (0,0) -- (90:0.25) arc [start angle=90,end angle={90+360/11},radius=0.25];
                \filldraw[fill=sgbGreen2,draw=black] (90:2.5) -- (90:2.25) arc [start angle=270,end angle={270-9*90/11},radius=0.25];
                \filldraw[fill=sgbGreen2,draw=black] ({90+360/11}:2.5) -- ({90+360/11}:2.25) arc [start angle={270+360/11},end angle={270++360/11+9*90/11},radius=0.25];
                }}

            \only<8>{
                \tikz{\node[regular polygon, regular polygon sides={11}, draw=sgbRed, minimum size=5cm] at (0,0) {};
                \foreach \angle in {0,...,10}{
                    \draw (0,0) -- ({90+\angle*360/11}:2.5);
                    }
                \filldraw[fill=sgbGreen2,draw=black] (0,0) -- (90:0.25) arc [start angle=90,end angle=450,radius=0.25];
                \foreach \angle in {0,...,10}{
                \filldraw[fill=sgbGreen2,draw=black] ({90+\angle*360/11}:2.5) -- ({90+\angle*360/11}:2.25) arc [start angle={270+\angle*360/11},end angle={270+\angle*360/11-9*90/11},radius=0.25];
                \filldraw[fill=sgbGreen2,draw=black] ({90+360/11+\angle*360/11}:2.5) -- ({90+360/11+\angle*360/11}:2.25) arc [start angle={270+\angle*360/11+360/11},end angle={270+\angle*360/11+360/11+9*90/11},radius=0.25];}}
                }
            
            \only<9>{
                \tikz{\node[regular polygon, regular polygon sides={11}, draw=sgbRed, minimum size=5cm] at (0,0) {};
                \foreach \angle in {0,...,10}{
                    \draw (0,0) -- ({90+\angle*360/11}:2.5);
                }
                \foreach \angle in {0,...,10}{
                    \filldraw[fill=sgbGreen2,draw=black] ({90+\angle*360/11}:2.5) -- ({90+\angle*360/11}:2.25) arc [start angle={270+\angle*360/11},end angle={270+\angle*360/11-9*90/11},radius=0.25];
                    \filldraw[fill=sgbGreen2,draw=black] ({90+360/11+\angle*360/11}:2.5) -- ({90+360/11+\angle*360/11}:2.25) arc [start angle={270+\angle*360/11+360/11},end angle={270+\angle*360/11+360/11+9*90/11},radius=0.25];
                    }}}
        
            \only<10>{
                \tikz{\node[regular polygon, regular polygon sides={15}, draw=sgbRed, minimum size=5cm] at (0,0) {$n$-gon};
                }}
    
            \only<11-12>{
                \tikz{\foreach \angle in {0,...,4}{
                        \draw (0,0) -- ({90+\angle*360/5}:2.5);}
                    \foreach \angle[count=\ai] in {0,...,2}{
                        \draw (0,0) ++ ({70+\angle*360/5}:2) node[left] {\ai};}
                        \draw (0,0) ++ ({70+3*360/5}:2) node[left] {$\ldots$};
                        \draw (0,0) ++ ({70+4*360/5}:2) node[left] {$m$};
                        }}
            \only<13>{
                \tikz{
                    \clip (0.1,0.1) rectangle (2.5,2.5);
                    \foreach \row in {0,...,3}{
                        \draw (0,{\row*sqrt(3)/2}) --++ (0:3);
                        \draw (\row,0) --++ (60:4);
                        \draw (\row,0) --++ (120:4);}}
                \tikz{
                \clip (0.1,0.1) rectangle (2.5,2.5);
                \draw (0,0) grid (3,3);}
                }
        
            \only<15-18>{
                \tikz{
                    \foreach \angle in {7, 15, 22, 29, 34}{
                        \draw (0,0) -- ({10*\angle}:2);
                    }
                    \foreach \angle[count=\ai] in {7, 15, 22}{
                        \node at ({10*\angle+20}:2) {$n_\ai$-gon};
                    }
                    \node[left] at ({10*29+20}:2) {\ldots};
                    \node[left] at ({10*34+20}:2) {$n_m$-gon};
                }
            }
        \end{column}
        \begin{column}{0.6\textwidth}
            \only<6-9>{for $n$ = 11, we have 11 triangles.}
            
            \only<7-10>{Sum of angles of one triangle = $\pi$
            
            $\pi = \ang{180}$}
            
            \only<8-9>{Sum of angles of 11 triangles = $11\pi$}
            
            \only<9>{Sum of internal angles = $11\pi - 2\pi$
            
            Each internal angle  = $\flatfrac{\qty(11\pi - 2\pi)}{11}$}
            
            \only<10>{when we have $n$ triangles,

            Sum of angles of $n$ triangles = $n\pi$

            Sum of internal angles = $n\pi - 2\pi$}
            
            \only<10-12,16-18>{
                Internal angle  = $\flatfrac{\qty(n - 2)\pi}{n} = A_n$
            }
            
            \only<11-12>{
                $m \times A_n = \ang{360} = 2\pi$
            }
            
            \only<13>{
                \tikzset{cell/.pic={
                \draw (0,{sqrt(3)/4})--++(0:0.25) --++(60:0.5)--++(0:0.5)--++(300:0.5)--++(0:0.25);
                \draw (0.25,{sqrt(3)/4})--++(300:0.5);
                \draw (1.25,{sqrt(3)/4})--++(240:0.5);
            }}
            \tikz{
            \clip (0.1,0.1) rectangle (4,3);
            \foreach \x in {0,...,3}{
                \foreach \y in {0,...,3}{
                    \draw ({1.5*\x},{\y*sqrt(3)/2}) pic {cell};
                }};}
            }
            
            \only<12-13>{
                \begin{equation*}
                \frac{1}{m} + \frac{1}{n} = \frac{1}{2}
                \end{equation*}
            }
            \only<16-18>{
                \begin{equation*}
                    A_{n_1}+A_{n_2}+A_{n_3}+\ldots+A_{n_m} = 2\pi
                \end{equation*}
            }
            \only<17->{
                \begin{equation*}
                    \frac{m}{2} - \qty(\frac{1}{n_1}+\frac{1}{n_2}+\frac{1}{n_3}+\ldots+\frac{1}{n_m})=1
                \end{equation*}
            }
            \only<19->{
                \begin{equation*}
                    \frac{3}{2} - \qty(\frac{1}{n_1}+\frac{1}{n_2}+\frac{1}{n_3})=1
                \end{equation*}
            }
        \end{column}
    \end{columns}    
\end{frame}

\begin{frame}
\end{frame}
\begin{frame}    
    \frametitle{Escher}
    Take from sswp
    \begin{columns}
        \begin{column}{0.45\textwidth}    
        \end{column}
        \begin{column}{0.45\textwidth}    
        \end{column}
    \end{columns}
\end{frame}
\begin{frame}
\frametitle{Wang's Conjecture}
\end{frame}
\begin{frame}
    \frametitle{Are there sets of tiles that tile only periodically?}
\end{frame}
\begin{frame}
Escher's ascending descending - penrose staircases
\end{frame}
\begin{frame}
Ratio of number of darts and kites
if the ratio was rational? - periodic Tiling
place the forced pieces first
when you have choices - one can lead to a point where no more pieces can be legally added
number of penrose tilings? - 'uncountable'
uncountable meaning? - e.g. rational numbers are countable, real numbers are uncountable

\end{frame}

\begin{frame}
\frametitle{Reference}
Vigyan Pratibha Learning Unit - \url{https://vigyanpratibha.in/}
\end{frame}

\end{document}

\begin{frame}
    \frametitle{Create a volume measuring instrument}
            \uncover<2>{
                Least Count
            }
\end{frame}
\begin{frame}
\frametitle{Measure the volume of solids}
\uncover<2>{
    Overflow principle vs. Archimedes principle

    Any object, whether fully or partially submerged in a fluid, is pushed up by a force equal to the weight of the fluid displaced by the object.

    For a submerged object, the volume of displaced fluid equals the volume of the object.
}
\end{frame}
\begin{frame}
    \frametitle{Back to the Thirsty Crow}
    \begin{columns}
        \begin{column}{0.35\textwidth}    
            %\includegraphics[height=0.55\textheight]{Crow}\footnote{Project Gutenberg}
        \end{column}
        \begin{column}{0.55\textwidth}    
                \begin{enumerate}
                    \item<2-> Does putting enough number of marbles always make the water level rise to the top of the container?
                    \item<3-> If you do not succeed in raising the water level to the top, can you estimate the maximum marking to
                    which the water level rises?
                    \item<4-> Explain your observation.
                \end{enumerate}
        \end{column}
    \end{columns}
\end{frame}
\begin{frame}
    \frametitle{Back to the Thirsty Crow}
    \begin{columns}
        \begin{column}{0.35\textwidth}    
            %\includegraphics[height=0.55\textheight]{Crow}\footnote{Project Gutenberg}
        \end{column}
        \begin{column}{0.55\textwidth}    
                \begin{enumerate}
                    \item[4] What happens if we use a bottle with a larger or smaller diameter, of if we use marbles that are larger or smaller? Does the packing fraction change?
                    % \item[5]<2-> 
                    % \item[6]<3-> 
                \end{enumerate}
        \end{column}
    \end{columns}
\end{frame}

\begin{frame}
\frametitle{Packing}
\uncover<2>{We have a dining table that is $80 \unit{\centi\metre}$ wide and $120 \unit{\centi\metre}$ long. Our round dinner plates have a radius of $10 \unit{\centi\metre}$. How many plates can we put on our dinner table without stacking?}

%\includegraphics[height=0.25\textheight]{cylinder}
%\includegraphics[height=0.25\textheight]{plate}

\end{frame}
\begin{frame}
    \frametitle{Packing in 3D space}
    %\includegraphics[height=0.25\textheight]{watermelon}
    %\includegraphics[height=0.25\textheight]{orange}
    %\includegraphics[height=0.25\textheight]{orange2}
    % %\includegraphics[height=0.75\textheight]{packing}
\end{frame}
\begin{frame}
\frametitle{Packing Fraction}
\begin{equation*}
\frac{\text{Volume occupied by solids}}{\text{Total volume}}
\end{equation*}
\end{frame}
\begin{frame}
% \frametitle{Figurate numbers}


\begin{frame}
        \begin{columns}
        \begin{column}{0.45\textwidth}
        \end{column}
        \begin{column}{0.45\textwidth}
        \end{column}
    \end{columns}
\end{frame}
\begin{frame}
\frametitle{Natural Pinhole Camera Effect}
\href{https://www.reddit.com/r/woahdude/comments/lzv1js/the_curtains_in_my_room_create_a_natural_pinhole}{\beamergotobutton{Link}}
% \uncover<2>{
% %\includegraphics[width=0.7\textwidth]{solar-eclipse-2010}\footnote{Picture taken by Guganeshan.T from Srilanka during the longest, annular Solar Eclipse of the millennium that happened on 15th January 2010.}}
\end{frame}
\begin{frame}
    \frametitle{Worldwide Pinhole Photography Day}
    \begin{block}{April 27, 2025}
        Last Sunday in April is celebrated as the annual Worldwide Pinhole Photography Day.
    \end{block}
    \uncover<2>{
        \begin{columns}
            \begin{column}{0.5\textwidth}
                Zehao Peng, "My University"
                Shenzhen, Guangdong, China
                % %\includegraphics[width=0.9\textwidth]{0611}\footnote{Copyright 2024 Zehao Peng}
            \end{column}
            \begin{column}{0.4\textwidth}
                % %\includegraphics[width=0.9\textwidth]{0611_setup}
            \end{column}
        \end{columns}
    }
\end{frame}
\begin{frame}
    \begin{columns}
        \begin{column}{0.5\textwidth}
        \end{column}
        \begin{column}{0.4\textwidth}
        \end{column}
    \end{columns}
\end{frame}

\begin{frame}
    \begin{columns}
        \begin{column}{0.5\textwidth}
        \end{column}
        \begin{column}{0.4\textwidth}
        \end{column}
    \end{columns}
\end{frame}
\begin{frame}
    \begin{center}
    \end{center}
    \begin{columns}
        \begin{column}{0.45\textwidth}    
        \end{column}
        \begin{column}{0.45\textwidth}    
        \end{column}
    \end{columns}
\end{frame}
\begin{frame}
    \begin{columns}
        \begin{column}{0.4\textwidth} 
        \end{column}
        \begin{column}{0.4\textwidth} 
        \end{column}
    \end{columns}
\end{frame}
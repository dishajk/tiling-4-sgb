\documentclass{beamer}
\mode<presentation>
\usepackage{hyperref}
\usepackage{pdfpages}
\usepackage{siunitx}

\title{Tiling}
\author[dishajk]{Disha Kuzhively\\ \texttt{disha.jk@icts.res.in}}
\institute{International Centre for Theoretical Sciences - TIFR, Bangalore}

\begin{document}
\begin{frame}
    \titlepage
\end{frame}
\begin{frame}
    \frametitle{Escher}
    Take from sswp
    \begin{columns}
        \begin{column}{0.45\textwidth}    
        \end{column}
        \begin{column}{0.45\textwidth}    
        \end{column}
    \end{columns}
\end{frame}
\begin{frame}
\frametitle{Wang's Conjecture}
\end{frame}
\begin{frame}
    \frametitle{Are there sets of tiles that tile only periodically?}
\end{frame}
\begin{frame}
Escher's ascending descending - penrose staircases
\end{frame}
\begin{frame}
Ratio of number of darts and kites
if the ratio was rational? - periodic Tiling
place the forced pieces first
when you have choices - one can lead to a point where no more pieces can be legally added
number of penrose tilings? - 'uncountable'
uncountable meaning? - e.g. rational numbers are countable, real numbers are uncountable

\end{frame}

\begin{frame}
\frametitle{Reference}
Vigyan Pratibha Learning Unit - \url{https://vigyanpratibha.in/}
\end{frame}

\end{document}

\begin{frame}
    \frametitle{Create a volume measuring instrument}
            \uncover<2>{
                Least Count
            }
\end{frame}
\begin{frame}
\frametitle{Measure the volume of solids}
\uncover<2>{
    Overflow principle vs. Archimedes principle

    Any object, whether fully or partially submerged in a fluid, is pushed up by a force equal to the weight of the fluid displaced by the object.

    For a submerged object, the volume of displaced fluid equals the volume of the object.
}
\end{frame}
\begin{frame}
    \frametitle{Back to the Thirsty Crow}
    \begin{columns}
        \begin{column}{0.35\textwidth}    
            %\includegraphics[height=0.55\textheight]{Crow}\footnote{Project Gutenberg}
        \end{column}
        \begin{column}{0.55\textwidth}    
                \begin{enumerate}
                    \item<2-> Does putting enough number of marbles always make the water level rise to the top of the container?
                    \item<3-> If you do not succeed in raising the water level to the top, can you estimate the maximum marking to
                    which the water level rises?
                    \item<4-> Explain your observation.
                \end{enumerate}
        \end{column}
    \end{columns}
\end{frame}
\begin{frame}
    \frametitle{Back to the Thirsty Crow}
    \begin{columns}
        \begin{column}{0.35\textwidth}    
            %\includegraphics[height=0.55\textheight]{Crow}\footnote{Project Gutenberg}
        \end{column}
        \begin{column}{0.55\textwidth}    
                \begin{enumerate}
                    \item[4] What happens if we use a bottle with a larger or smaller diameter, of if we use marbles that are larger or smaller? Does the packing fraction change?
                    % \item[5]<2-> 
                    % \item[6]<3-> 
                \end{enumerate}
        \end{column}
    \end{columns}
\end{frame}

\begin{frame}
\frametitle{Packing}
\uncover<2>{We have a dining table that is $80 \unit{\centi\metre}$ wide and $120 \unit{\centi\metre}$ long. Our round dinner plates have a radius of $10 \unit{\centi\metre}$. How many plates can we put on our dinner table without stacking?}

%\includegraphics[height=0.25\textheight]{cylinder}
%\includegraphics[height=0.25\textheight]{plate}

\end{frame}
\begin{frame}
    \frametitle{Packing in 3D space}
    %\includegraphics[height=0.25\textheight]{watermelon}
    %\includegraphics[height=0.25\textheight]{orange}
    %\includegraphics[height=0.25\textheight]{orange2}
    % %\includegraphics[height=0.75\textheight]{packing}
\end{frame}
\begin{frame}
\frametitle{Packing Fraction}
\begin{equation*}
\frac{\text{Volume occupied by solids}}{\text{Total volume}}
\end{equation*}
\end{frame}
\begin{frame}
% \frametitle{Figurate numbers}


\begin{frame}
        \begin{columns}
        \begin{column}{0.45\textwidth}
        \end{column}
        \begin{column}{0.45\textwidth}
        \end{column}
    \end{columns}
\end{frame}
\begin{frame}
    \frametitle{Make a Pinhole Camera}
    \begin{columns}
        \begin{column}{0.45\textwidth}
            \begin{block}{Step 3}
                Cover one end of the tube with the marked tracing sheet. Let us call this the \alert{image tube}.
            \end{block}
        \end{column}
        \begin{column}{0.45\textwidth}
            \begin{block}{Step 4}
                Roll another sheet of black chart paper into a cylindrical tube such that it is slightly larger in diameter than the \alert{image tube}, so that the \alert{image tube} can slide inside smoothly.
            \end{block}
        \end{column}
    \end{columns}
\end{frame}
\begin{frame}
    \frametitle{Make a Pinhole Camera}
    \begin{columns}
        \begin{column}{0.45\textwidth}
            \begin{block}{Step 5}
                Cover one end of the secnd tube with aluminium foil. Let us call this the \alert{pinhole tube}. Poke a hole on the aluminium foil using a safety pin.
            \end{block}
        \end{column}
        \begin{column}{0.45\textwidth}
            \begin{block}{Step 6}
                Insert \alert{image tube} inside \alert{pinhole tube} till the screen just touches the pinhole. Mark apoint $O$ on the \alert{image tube} where \alert{pinhole tube} ends. This is to help measure the distance between the screen and pinhole($l$).
            \end{block}
        \end{column}
    \end{columns}
\end{frame}
\begin{frame}
% %\includegraphics[width=0.75\textwidth]{schematic}\footnote{Vigyan Pratibha Learning Unit}

\uncover<2>{
    What has changed from the object to the image?

    Change the following and see what happens.
    \begin{enumerate}
    \item Distance between the pinhole and the screen.
    \item Brightness of the object
    \item Larger pinhole size
    \end{enumerate}
    }
\end{frame}
\begin{frame}
\frametitle{Natural Pinhole Camera Effect}
\href{https://www.reddit.com/r/woahdude/comments/lzv1js/the_curtains_in_my_room_create_a_natural_pinhole}{\beamergotobutton{Link}}
% \uncover<2>{
% %\includegraphics[width=0.7\textwidth]{solar-eclipse-2010}\footnote{Picture taken by Guganeshan.T from Srilanka during the longest, annular Solar Eclipse of the millennium that happened on 15th January 2010.}}
\end{frame}
\begin{frame}
    \frametitle{Worldwide Pinhole Photography Day}
    \begin{block}{April 27, 2025}
        Last Sunday in April is celebrated as the annual Worldwide Pinhole Photography Day.
    \end{block}
    \uncover<2>{
        \begin{columns}
            \begin{column}{0.5\textwidth}
                Zehao Peng, "My University"
                Shenzhen, Guangdong, China
                % %\includegraphics[width=0.9\textwidth]{0611}\footnote{Copyright 2024 Zehao Peng}
            \end{column}
            \begin{column}{0.4\textwidth}
                % %\includegraphics[width=0.9\textwidth]{0611_setup}
            \end{column}
        \end{columns}
    }
\end{frame}
\begin{frame}
    \begin{columns}
        \begin{column}{0.5\textwidth}
        \end{column}
        \begin{column}{0.4\textwidth}
        \end{column}
    \end{columns}
\end{frame}

\begin{frame}
    \begin{columns}
        \begin{column}{0.5\textwidth}
        \end{column}
        \begin{column}{0.4\textwidth}
        \end{column}
    \end{columns}
\end{frame}
\begin{frame}
    \begin{center}
    \end{center}
    \begin{columns}
        \begin{column}{0.45\textwidth}    
        \end{column}
        \begin{column}{0.45\textwidth}    
        \end{column}
    \end{columns}
\end{frame}
\begin{frame}
    \begin{columns}
        \begin{column}{0.4\textwidth} 
        \end{column}
        \begin{column}{0.4\textwidth} 
        \end{column}
    \end{columns}
\end{frame}